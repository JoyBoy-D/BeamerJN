\usepackage[portuges]{babel}
% Para aceitar caracteres especias deretamente do teclado
\usepackage[utf8]{inputenc}

% Para incluir figuras
\usepackage{graphicx}
\usepackage{wrapfig}
\usepackage{epstopdf}
% \epstopdfDeclareGraphicsRule{.png}{pdf}{.pdf}{convert #1 \OutputFile}
\DeclareGraphicsExtensions{.png,.jpg,.pdf}
\graphicspath{{./images/} {./core/img/}}

% Para melhor ajuste da posisao das figuras
\usepackage{float}

% Para ajustar as dimensoes do layout da pagina
\usepackage{geometry}
% Para formatar estrutura e informacoes de formulas matematicas
\usepackage{amsmath}
% Para incluir simbolos especiais em formulas matematicas
\usepackage{amssymb}
% Para incluir links nas referencias
\usepackage{url}
% Para incluir paginas de documentos .pdf externos
\usepackage{pgfpages}
% Para ajustar o estilo dos contadores
\usepackage{enumerate}
% Para modificar a cor do texto
\usepackage{color}
% Para incluir condicoes
\usepackage{ifthen}
% Para colocar legendas em algo que nao e float
\usepackage{capt-of}

% Título
\title[Disciplina]{Conteúdo da apresentação}
% Data
\date{ \today}
% Autores
\author[Nome]{Nome Completo}

% Para seguir as normas da ABNT de citacao e referencias
\usepackage[alf]{abntex2cite}